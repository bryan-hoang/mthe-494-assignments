\documentclass{tech-spec}

\begin{document}
  \chapter{Technical Specification - The Right Way}

  \stepcounter{section}
  \section*{Elements of the Law}

  \begin{parnumbers}
    The law has three main elements: legislation, regulations, and precedents.

    The first element of the law is legislation. Legislation codifies societal rules. Legislation is also referred to as “statutes” and “ordinances.”

    Legislation provides the framework for the application of legal and regulatory standards.

    The second element of the law is regulations. Regulations provide additional rules and clarification to legislation. Regulations can assign standards created by government agencies and accredited standards organizations. Regulations give standards created by government agencies and accredited standards organizations legal force.

    The third element of the law is precedent. Precedents are rulings by judges in prior cases. Precedents are used as examples when deciding cases. Precedents are also referred to as “case law.”

    Legislation and regulations are divided into three jurisdictions: federal, provincial, and municipal.

    Federal legislation and regulations are created by the federal government. Federal legislation and regulations apply to all of Canada.

    Provincial legislation and regulations are created by a provincial or territorial government. Provincial legislation and regulations apply only within that province or territory.

    Municipal legislation and regulations are created by a municipal government. Municipal legislation and regulations apply only within that municipality.
  \end{parnumbers}

  \stepcounter{section}
  \section*{Legal and Regulatory Standards}
  \begin{parnumbers}
    Legal and regulatory standards apply to: products, processes, and services.

    Legal and regulatory standards can define: quality, performance, safety, dimensions, and/or labelling.

    Legal and regulatory standards are governed by: legislation, regulations, and standards created by government agencies or accredited standards organizations.

    Non-government standards organizations are accredited by the federal and/or provincial government. Federal accreditation is administered by the Standards Council of Canada. An example of an accredited organization is the Canadian Standards Association (CSA). Standards apply to the jurisdiction of the legislation and/or regulation(s) giving them legal force.
  \end{parnumbers}

\end{document}
