\documentclass{tech-spec}

\begin{document}
  \chapter{Technical Specification - Multi-wheel heeling apparatus (Heelys)}

  \stepcounter{section}
  \section*{Soles and Openings}
  \begin{parnumbers}
    A multi-wheel heeling apparatus consists of 3 portions: a forefoot portion, a heel portion, and an arch portion.

    The forefoot portion of the sole is inoperable for rolling to provide primary contact with the surface for: walking, running, and inhibit rolling.

    The heel portion of the sole of the footwear includes a brake operable for slowing the heeling apparatus.

    A multi-wheel heeling apparatus has two openings: a first opening, and a second opening.

    The first opening is formed in a heel portion of the sole.

    The second opening is formed in at least a portion of the arch portion of the sole adjacent the first opening.
  \end{parnumbers}

  \stepcounter{section}
  \section*{Wheel assemblies}
  \begin{parnumbers}
    The multi-wheel heeling apparatus includes: a first wheel assembly, and a second wheel assembly.

    The first wheel assembly includes a first wheel mounted on a first axle. The first wheel assembly includes a first mounting structure operable to support the first axle so that portion of the first wheel resides in the first opening.

    The first wheel assembly is removably coupled to the wheel mounting structure.

    The second wheel assembly includes a second wheel mounted on a second axle. The second wheel assembly includes a second mounting structure operable to support the second axle so that a portion of the second wheel resides in the second opening.

    The second wheel assembly is removably coupled to the wheel mounting structure.

    The first axle and second axles tensioningly couple to the wheel mounting structure.
  \end{parnumbers}

  \stepcounter{section}
  \section*{Wheels}
  \begin{parnumbers}
    A wheel may be constructed or made of virtually any known or available material such as: a urethane, a plastic, a polymer, a metal, an alloy, a wood, a rubber, and a composite material. The material a wheel is made of may include: aluminum, titanium, steel, and a resin.

    The material a wheel is made of will: be durable, provide quiet performance, and will provide a “soft” or “cushioning” feel.

    The amount or length of the portion of a wheel that extends below the bottom of the sole will preferably be less than the diameter of the wheel. The amount or length of the portion of a wheel that extends below the bottom of the sole may not be restricted by the diameter of the wheel.

    The diameter of the first wheel is substantially similar to a diameter of the second wheel.

    The first and second wheels are removably coupled to the mounting structure.

    The first and second wheels provide contact with the surface to roll on the surface when the forefoot is disengaged from the surface.
  \end{parnumbers}

  \stepcounter{section}
  \section*{Applicable Standards}
  \begin{parnumbers}
    A multi-wheel heeling apparatus is governed by all the applicable standards contained in ICS 61.060 (Footwear Including shoelaces).
  \end{parnumbers}
\end{document}
