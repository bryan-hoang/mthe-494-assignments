\documentclass[
  10pt,
  promotesection,
  endnotes,
  draft,
  % final,
]{memreport}

\usepackage{bch-style}

\title{
  MTHE 494 Cat C:\@ Boeing 737 MAX 8 Crash\\
  Lion Air Flight 610
}
\author{Bryan Hoang}
\date{\today}

\addbibresource{\main/bibliography.bib}

\begin{document}
  \begin{titlingpage}
    \maketitle
  \end{titlingpage}

  \mainmatter{}

  % Normalizing date style.
  \DTMlangsetup[en-US]{ord=raise}
  \DTMsetdatestyle{en-US}

  \section{Introduction}
  \subsection{Background}
  Boeing 737--8 (MAX) aircraft is the successor of the Boeing 737 NG. One of the design changes from the Boeing 737 NG is a Maneuvering Characteristics Augmentation System (MCAS) that aims to improve aircraft handling characteristics at elevated angles of attack\autocite[I]{noauthor_boeing_2019}.

  The Boeing 737--8 (MAX) is also equipped with two Angle of Attack (AOA) sensors that each measure indicated airspeed (IAS), altitude, and angle of attack\autocite[xviii]{noauthor_aircraft_2019}. MCAS only relies on the left AOA sensor when determining if the aircraft's AOA will result in a stall\autocite[\nopp{}195]{noauthor_aircraft_2019}. In that event, it will automatically lower the stabilizers to decrease the aircraft's AOA.\@

  \subsection{Chronology of Events}
  On \DTMdisplaydate{2018}{10}{26}{-1}, the SPD (speed) and ALT (altimeter) flags on the Captain's primary flight display had problems during a flight from Tianjin, China to Manado, Indonesia. Following reoccurrence of these problems, the left angle of attack (AOA) sensor was replaced in Denpasar on \DTMdisplaydate{2018}{10}{28}{-1}\autocite[\nopp{}xviii]{noauthor_aircraft_2019}.

  On \DTMdisplaydate{2018}{10}{29}{-1} at 23:15:00 UTC, Lion Mentari Airlines (Lion Air) flight 610 Boeing 737--8 (MAX) began departing from Soekarno-Hatta International Airport, Jakarta to Depati Amir Airport, Pangkal Pinang. The flight had 181 passengers and 8 crew members on board\autocite[\nopp{}19]{noauthor_aircraft_2019}. At 23:20:16 UTC, the Digital Flight Data Recorder (DFDR) recorded a difference angle between left and right Angle of Attack (AOA) sensors, where the left AOA sensor was about 21° higher. The discrepancy continued until the end of recording\autocite[\nopp{}19]{noauthor_aircraft_2019}.

  At 23:20:35, as the nose gear lifted off the runway, the left control column stick shaker activated which continued for most of the flight. At 23:20:51, the landing gear are moved up. Then First Officer (FO) and Captain notice that both their indicated airspeed and altitude readings disagree. At 23:21:28, the FO asks the Terminal East (TE) controller to confirm the aircraft altitude, to which the TE controller responds that the aircraft altitude was 900\autocite[\nopp{}20]{noauthor_aircraft_2019}.

  At 23:21:52, unsure of their current airspeed and altitude, the FO requested clearance from the TE controller ``to some holding point for our condition now''. The TE controller asked the LNI610, what was the problem of the aircraft and the FO responded ``flight control problem'' \autocite[\nopp{}21]{noauthor_aircraft_2019}. At 23:25:05, the FO begins going through the checklist for air speed unreliable for three minutes. The Captain and FO turn off the autopilot and continue to try to troubleshoot the problem, extending and retracting the flaps, during which the automatic Aircraft Nose Down (AND) trim activated intermittently when the flaps were fully retracted. The captain applied manual Aircraft Node Up (trim) commands to counteract the automatic ANU trim commands\autocite[\nopp{}22-26]{noauthor_aircraft_2019}.

  At 23:29:37 UTC, the TE controller noticed that the aircraft was descending on the radar screen. At 23:30:02, the FO contacted the Arrival (ARR) controller, who advised the LNI610 flight crew to prepare for landing on runway 25L. At 23:30:48, the Captain asked the FO to take over control of the aircraft\autocite[\nopp{}26]{noauthor_aircraft_2019}. At 23:31:46, MCAS activated again and the rate of descent of the aircraft was more than \num{10 000} feet/minute. At 23:31:53, The aircraft impacted the water in Tanjung Karawang, West Java, killing all 189 people on board\autocite[\nopp{}27]{noauthor_aircraft_2019}.

  \section{Analysis}
  \subsection{General}
  The aircraft had a valid airworthiness certificate. But, the plane's left AOA sensor was faulty when flight 610 took off. Therefore, it could not have been deemed airworthy at the time of the flight.

  The flight crew had a combined 11202 flying hours, with 9462 combined hours on the Boeing 737. Both of their certifications were up to date, but the FO had a history of difficulties with basic aircraft flight control during training and certification.

  \subsection{Errors and Violations}
  The (maintenance) ground crew that installed left AOA sensor with a 21° bias  erroneously did not detect the bias during the installation test in Denpasar\autocite[\nopp{}xviii]{noauthor_aircraft_2019}. The issue was thus present during both of the subsequent flights.

  The crew from the flight before flight 610 failed to report on some malfunctions present on the aircraft, such as the activation of the stick shaker and STAB TRIM to CUT OUT\autocite[\nopp{}xviii]{noauthor_aircraft_2019}. That crew's failure to follow procedure on reporting every problem properly led to the crew of flight 610 being incorrectly informed of the aircraft's malfunctioning left AOA sensor.

  When the Captain made the FO take control, he mistakenly did not inform the FO of the need to continuously make ANU commands to the stabilizer to keep the aircraft's nose up. This, combined with the FO's history of inadequate manual flight control skills, led to a lack of correcting the MCAS' AND trim commands ultimately leading to the crash.

  \subsection{Resident Pathogens}
  Boeing made the AOA DISAGREE alert optional for the 737--8 (MAX) aircraft as they did not consider it a necessary safety feature. Lion Air did not select the optional AOA indicator feature on the PFD of their 737--8 (MAX) aircraft. As a result, the AOA DISAGREE did not appear on PK-LQP aircraft, even though the necessary conditions were met\autocite[\nopp{}45--46]{noauthor_aircraft_2019}. Since the AOA DISAGREE alert was not available on the aircraft, the previous flight crew did not feel the need to report the activation of stick shaker and STAB TRIM to CUT OUT\autocite[\nopp{}xviii]{noauthor_aircraft_2019}. If the alert was in place, the crew would have been more likely to report the error to get it rectified.

  Boeing implemented the MCAS without mentioning it in the aircraft's Flight Crew Operations Manual (FCOM) and without requiring additional pilot differences training in the simulator. As such, the flight crew's reactions were different from and did not match the guidance for assumptions of flight crew behaviour that were used when classifying the hazard severity of the MCAS' failure mode in the functional hazard assessment. The MCAS function was also not a fail-safe design and did not include redundancy, since a single failure to the left AOA sensor resulted in erroneous activation of MCAS throughout the flight\autocite[\nopp{}198--200]{noauthor_aircraft_2019}.

  \subsection{Triple Bottom Line}
  Boeing designed and treated the MCAS as a way to circumvent the FAA's certification process. The company downplayed the MCAS' effect on aircraft handling, excluded it from the FCOM, made the AOA DISAGREE alert optional, and made MCAS rely only on one sensor in order to save time and money. Because of Boeing's greed to prioritize money, they endangered the public's safety.

  Due to the lack of documentation and training with the MCAS system, as well as failure of aircraft's components, the flight crew was misinformed and unprepared, leading to the in the death of 189 individuals. Those individuals' families and communities will forever be impacted by the unethical steps taken by Boeing's desire to compete with AirBus. The destroyed aircraft from the crash incurred cleanup costs and polluted the sea, thus harming its environment.

  \section{Recommendations and Conclusion}
  It's recommended that the FAA's modify their certification process to require that all US type-certificated transport-category airplane manufacturers ensure that system safety assessments for the Boeing 737--8 (MAX) in which it assumed immediate and appropriate pilot corrective actions in response to uncommanded flight control inputs, from systems such as the MCAS, consider the effect of all possible flight deck alerts and indications on pilot recognition and response. It's also recommended manufacturers improve design enhancements (including flight deck alerts and indications), pilot procedures, and/or training requirements, where needed, to minimize the potential for and safety impact of pilot actions that are inconsistent with manufacturer assumptions. Lastly, it's recommended that the FAA develop robust tools and methods, with the input of industry and human factors experts, for use in validating assumptions about pilot recognition and response to safety-significant failure conditions as part of the design certification process\autocite[12]{noauthor_safety_2019}.

  \backmatter{}
\end{document}
